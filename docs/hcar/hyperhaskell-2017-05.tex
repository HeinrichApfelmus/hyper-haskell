% HyperHaskellThestronglyhy-HH.tex
\begin{hcarentry}[updated]{HyperHaskell -- The strongly hyped Haskell interpreter}
\label{hyper-haskell}
\report{Heinrich Apfelmus}%05/17
\release{0.1.0.0}
\status{available, active development}
\makeheader

\emph{HyperHaskell} is a graphical Haskell interpreter, not unlike GHCi, but
hopefully more awesome. You use worksheets to enter expressions and evaluate
them. Results are displayed graphically using HTML.

\emph{HyperHaskell} is intended to be \emph{easy to install}. It is
cross-platform and should run on Linux, Mac and Windows. Internally, it uses
the GHC API to interpret Haskell programs, and the graphical front-end is
built on the Electron framework. \emph{HyperHaskell} is open source.

\emph{HyperHaskell}'s main attraction is a \verb`Display` class that
supersedes the good old \verb`Show` class. The result looks like this:

%**<img width=700 src="./worksheet-diagrams.png">
%*ignore
\begin{center}
  \includegraphics[width=\columnwidth]{html/worksheet-diagrams.png}
\end{center}
%*endignore

\subsubsection*{Current status}

\emph{HyperHaskell} is currently \emph{Level $\alpha$}. Compared to the previous report, no new release has been made, but basic features are working. A new cell type, the \emph{text cell}, has been implemented, but not yet released.

I am looking for help in setting up binary releases on the Windows platform!

\subsubsection*{Future development}

Programming a computer usually involves writing a program text in a particular
language, a ``verbal'' activity. But computers can also be instructed by
gestures, say, a mouse click, which is a ``nonverbal'' activity. The long term
goal of \emph{HyperHaskell} is to blur the lines between programming
``verbally'' and ``nonverbally'' in Haskell. This begins with an interpreter
that has graphical representations for values, but also includes editing a
program text while it's running (``live coding'') and interactive
representations of values (e.g. ``tangible values''). This territory is still
largely uncharted from a purely functional perspective, probably due to a lack
of easily installed graphical facilities. It is my hope that
\emph{HyperHaskell} may provide a common ground for exploration and
experimentation in this direction, in particular by offering the
\verb!Display! class which may, perhaps one day, replace our good old
\verb!Show! class.

A simple form of live coding is planned for \emph{Level $\beta$}.

\FurtherReading
\begin{compactitem}
  \item Project homepage and downloads: \url{https://github.com/HeinrichApfelmus/hyper-haskell}
\end{compactitem}
\end{hcarentry}
